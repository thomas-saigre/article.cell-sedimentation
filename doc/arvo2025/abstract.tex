\documentclass[a4paper]{article}
\usepackage{fullpage}
\usepackage{graphicx}
\usepackage{xcolor}
\usepackage{subcaption}
\usepackage{booktabs}
\usepackage{tabularray}
\usepackage{multirow}
\usepackage{pgfplots}
\pgfplotsset{compat=newest}

\usepackage{siunitx}
\DeclareSIUnit\mmHg{mmHg}

\usepackage{hyperref}
\usepackage{cleveref}
\title{Template Repository for Articles}
\author{
    Christophe Prud'homme\footnotemark[1],
    Thomas Saigre\footnotemark[1],
    Sangly P. Srinivas\footnotemark[2],
    Marcela Szopos\footnotemark[3]
}

\usepackage{csquotes}
\usepackage[natbib, defernumbers=true, backend=biber, style=alphabetic, eprint=false, maxbibnames=99]{biblatex}
\addbibresource{biblio.bib}

\makeatletter
\newcommand\addplotgraphicsnatural[2][]{%
    \begingroup
    % set options in this local group (will be lost afterwards):
    \pgfqkeys{/pgfplots/plot graphics}{#1}%
    % measure the natural size of the graphics:
    \setbox0=\hbox{\includegraphics{#2}}%
    %
    % compute the required unit vector ratio:
    \pgfmathparse{\wd0/(\pgfkeysvalueof{/pgfplots/plot graphics/xmax} - \pgfkeysvalueof{/pgfplots/plot graphics/xmin})}%
    \let\xunit=\pgfmathresult
    \pgfmathparse{\ht0/(\pgfkeysvalueof{/pgfplots/plot graphics/ymax} - \pgfkeysvalueof{/pgfplots/plot graphics/ymin})}%
    \let\yunit=\pgfmathresult
    %
    % configure pgfplots to use it.
    % The \xdef expands all macros except those prefixed by '\noexpand'
    % and assigns the result to a global macro named '\marshal'.
    \xdef\marshal{%
        \noexpand\pgfplotsset{unit vector ratio={\xunit\space \yunit}}%
    }%
    \endgroup
    %
    % use our macro here:
    \marshal
    %
    \addplot graphics[#1] {#2};
}
\makeatother


\begin{document}
\renewcommand{\thefootnote}{\fnsymbol{footnote}} % Use symbols globally


\maketitle

\footnotetext[1]{Cemosis, IRMA UMR 7501, Université de Strasbourg, CNRS, France}
\footnotetext[2]{Indiana University Bloomington, Bloomington, IN, USA}
\footnotetext[3]{Université Paris Cité, CNRS, MAP5, F-75006 Paris, France}

\section*{Purpose}



\section*{Methods}





\section*{Results}




\begin{figure}
    \centering

    \def\subfigwidth{\textwidth}

    \begin{subtable}{0.5\textwidth}
        \includegraphics[width=\textwidth]{tab_validation}
        \caption{Comparison of the magnitude of the WSS results with results from the literature.}
        \label{tab:validation}
    \end{subtable}
    %
    \begin{subtable}{0.45\textwidth}
        \centering
        \includegraphics[width=\textwidth]{tab_wss}
        \caption{Mean of the magnitude of the WSS for different zones of the eye.}
        \label{tab:wss}
    \end{subtable}

    \begin{subfigure}{0.49\textwidth}
        \centering
        \begin{tikzpicture}
            \begin{axis}[
                colorbar,
                colormap/jet, % Choose the colormap you prefer
                % axis equal image,
                enlargelimits=false,
                colorbar horizontal,
                point meta min=15.417566299438477,
                point meta max=15.585275650024414,
                axis line style = {draw=none},
                tick style = {draw=none},
                xtick = \empty, ytick = \empty,
                colorbar style={
                    % xlabel style={
                    %     at={(0.5,1.1)},
                    %     anchor=south,
                    % },
                    xlabel = {$p$ [\si{\mmHg}]},
                    height=0.05*\pgfkeysvalueof{/pgfplots/parent axis height},
                    width=0.9*\pgfkeysvalueof{/pgfplots/parent axis width},
                    at={(0.5,-0.02)},
                    anchor=center,
                    tick label style={font=\footnotesize},
                },
                colorbar/draw/.append code={
                    \begin{axis}[
                        colormap={Gray and Red}{
                            rgb255(-1cm)=(26,26,26);
                            rgb255(-0.87451cm)=(58,58,58);
                            rgb255(-0.74902cm)=(91,91,91);
                            rgb255(-0.623529cm)=(128,128,128);
                            rgb255(-0.498039cm)=(161,161,161);
                            rgb255(-0.372549cm)=(191,191,191);
                            rgb255(-0.247059cm)=(215,215,215);
                            rgb255(-0.121569cm)=(236,236,236);
                            rgb255(0.00392157cm)=(254,254,253);
                            rgb255(0.129412cm)=(253,231,218);
                            rgb255(0.254902cm)=(250,204,180);
                            rgb255(0.380392cm)=(244,170,136);
                            rgb255(0.505882cm)=(228,128,101);
                            rgb255(0.631373cm)=(208,84,71);
                            rgb255(0.756863cm)=(185,39,50);
                            rgb255(0.882353cm)=(147,14,38);
                            rgb255(1cm)=(103,0,31);
                        },
                        colorbar horizontal,
                        point meta min=1.6125707421871079e-9,
                        point meta max=0.000007417459292689545,
                        every colorbar,
                        anchor=center,
                        colorbar shift,
                        colorbar=false,
                        xlabel = {$\vec{u}$ [\si{\meter\per\second}]},
                        height=0.05*\pgfkeysvalueof{/pgfplots/parent axis height},
                        width=0.9*\pgfkeysvalueof{/pgfplots/parent axis width},
                        at={(0.5,-0.1*3.5*\pgfkeysvalueof{/pgfplots/parent axis height})},
                        tick label style={font=\footnotesize},
                    ]
                        \pgfkeysvalueof{/pgfplots/colorbar addplot}
                    \end{axis}
                    },
                width=\subfigwidth
            ]
                \addplotgraphicsnatural[xmin=0, xmax=1, ymin=0, ymax=1]{flowsupine.png}

                \draw[->] (0.1, 0.88) -- (0.2, 0.88) node[midway, anchor=south] {$\vec{g}$};

                \draw[->, red] (0.1, 0.1) -- (0.2, 0.1) node[pos=1, anchor=north] {$x$};
                \draw[->, green!60!black] (0.1, 0.1) -- (0.1, 0.2) node[pos=1, anchor=east] {$y$};
                \draw[blue] (0.1, 0.1) node {$\odot$};
                \draw[blue] (0.1, 0.1) node[anchor=north east] {$z$};
            \end{axis}
        \end{tikzpicture}
        \caption{Velocity and pressure simulated for a subject in supine position.}
        \label{fig:supine}
    \end{subfigure}
    %
    \begin{subfigure}{0.49\textwidth}
        \centering
        \begin{tikzpicture}
            \begin{axis}[
                colorbar,
                colormap/jet, % Choose the colormap you prefer
                % axis equal image,
                enlargelimits=false,
                colorbar horizontal,
                point meta min=15.414600372314453,
                point meta max=15.582489967346191,
                axis line style = {draw=none},
                tick style = {draw=none},
                xtick = \empty, ytick = \empty,
                colorbar style={
                    % xlabel style={
                    %     at={(0.5,1.1)},
                    %     anchor=south,
                    % },
                    xlabel = {$p$ [\si{\mmHg}]},
                    height=0.05*\pgfkeysvalueof{/pgfplots/parent axis height},
                    width=0.9*\pgfkeysvalueof{/pgfplots/parent axis width},
                    at={(0.5,-0.02)},
                    anchor=center,
                    tick label style={font=\footnotesize},
                },
                colorbar/draw/.append code={
                    \begin{axis}[
                        colormap={Gray and Red}{
                            rgb255(-1cm)=(26,26,26);
                            rgb255(-0.87451cm)=(58,58,58);
                            rgb255(-0.74902cm)=(91,91,91);
                            rgb255(-0.623529cm)=(128,128,128);
                            rgb255(-0.498039cm)=(161,161,161);
                            rgb255(-0.372549cm)=(191,191,191);
                            rgb255(-0.247059cm)=(215,215,215);
                            rgb255(-0.121569cm)=(236,236,236);
                            rgb255(0.00392157cm)=(254,254,253);
                            rgb255(0.129412cm)=(253,231,218);
                            rgb255(0.254902cm)=(250,204,180);
                            rgb255(0.380392cm)=(244,170,136);
                            rgb255(0.505882cm)=(228,128,101);
                            rgb255(0.631373cm)=(208,84,71);
                            rgb255(0.756863cm)=(185,39,50);
                            rgb255(0.882353cm)=(147,14,38);
                            rgb255(1cm)=(103,0,31);
                        },
                        colorbar horizontal,
                        point meta min=0,
                        point meta max=0.0000062121127024231835,
                        every colorbar,
                        anchor=center,
                        colorbar shift,
                        colorbar=false,
                        xlabel = {$\vec{u}$ [\si{\meter\per\second}]},
                        height=0.05*\pgfkeysvalueof{/pgfplots/parent axis height},
                        width=0.9*\pgfkeysvalueof{/pgfplots/parent axis width},
                        at={(0.5,-0.1*3.5*\pgfkeysvalueof{/pgfplots/parent axis height})},
                        tick label style={font=\footnotesize},
                    ]
                        \pgfkeysvalueof{/pgfplots/colorbar addplot}
                    \end{axis}
                    },
                width=\subfigwidth
            ]
                \addplotgraphicsnatural[xmin=0, xmax=1, ymin=0, ymax=1]{flowprone.png}

                \draw[->] (0.2, 0.88) -- (0.1, 0.88) node[midway, anchor=south] {$\vec{g}$};

                \draw[->, red] (0.1, 0.1) -- (0.2, 0.1) node[pos=1, anchor=north] {$x$};
                \draw[->, green!60!black] (0.1, 0.1) -- (0.1, 0.2) node[pos=1, anchor=east] {$y$};
                \draw[blue] (0.1, 0.1) node {$\odot$};
                \draw[blue] (0.1, 0.1) node[anchor=north east] {$z$};
            \end{axis}
        \end{tikzpicture}
        \caption{Velocity and pressure simulated for a subject in prone position.}
        \label{fig:prone}
    \end{subfigure}
\end{figure}


\section*{Conclusions}



\end{document}