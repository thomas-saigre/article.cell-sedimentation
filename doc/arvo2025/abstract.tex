\documentclass[a4paper]{article}
\usepackage[utf8]{inputenc}
\usepackage[T1]{fontenc}
\usepackage{amsmath, amssymb}
\usepackage{fullpage}
\usepackage{graphicx}
\usepackage{xcolor}
\usepackage{subfigure}
\usepackage{booktabs}
\usepackage{tabularray}
\usepackage{multirow}
\usepackage{paralist}
\usepackage{pgfplots}
\pgfplotsset{compat=newest}

\usepackage[english]{babel}

\usepackage{siunitx}
\DeclareSIUnit\mmHg{mmHg}
\newcommand{\fpp}{Feel\nolinebreak\hspace{-.05em}\raisebox{.4ex}{\tiny\bf +}\nolinebreak\hspace{-.10em}\raisebox{.4ex}{\tiny\bf +}}


\usepackage{hyperref}
\usepackage[capitalise]{cleveref}
\title{Effect of Cooling of the Ocular Surface on Endothelial Cell Sedimentation in Cell Injection Therapy: Insights from Computational Fluid Dynamics}
\author{
    Thomas Saigre\footnotemark[1],
    Vincent Chabannes\footnotemark[1],
    Giovanna Guidoboni\footnotemark[2],\\
    Christophe Prud'homme\footnotemark[1],
    Marcela Szopos\footnotemark[3],
    Sangly P. Srinivas\footnotemark[4]
}

% \usepackage{csquotes}
% \usepackage[natbib, defernumbers=true, backend=biber, style=alphabetic, eprint=false, maxbibnames=99]{biblatex}
% \addbibresource{biblio.bib}

\makeatletter
\newcommand\addplotgraphicsnatural[2][]{%
    \begingroup
    % set options in this local group (will be lost afterwards):
    \pgfqkeys{/pgfplots/plot graphics}{#1}%
    % measure the natural size of the graphics:
    \setbox0=\hbox{\includegraphics{#2}}%
    %
    % compute the required unit vector ratio:
    \pgfmathparse{\wd0/(\pgfkeysvalueof{/pgfplots/plot graphics/xmax} - \pgfkeysvalueof{/pgfplots/plot graphics/xmin})}%
    \let\xunit=\pgfmathresult
    \pgfmathparse{\ht0/(\pgfkeysvalueof{/pgfplots/plot graphics/ymax} - \pgfkeysvalueof{/pgfplots/plot graphics/ymin})}%
    \let\yunit=\pgfmathresult
    %
    % configure pgfplots to use it.
    % The \xdef expands all macros except those prefixed by '\noexpand'
    % and assigns the result to a global macro named '\marshal'.
    \xdef\marshal{%
        \noexpand\pgfplotsset{unit vector ratio={\xunit\space \yunit}}%
    }%
    \endgroup
    %
    % use our macro here:
    \marshal
    %
    \addplot graphics[#1] {#2};
}
\makeatother


\newcommand{\figwss}[7][west]{%
\vspace{-0.5cm}
\begin{tikzpicture}
\begin{axis}[
    colorbar,
    colormap={Rainbow Desaturated}{
        rgb255(0cm)=(70,70,219);
        rgb255(0.143cm)=(0,0,91);
        rgb255(0.285cm)=(0,255,255);
        rgb255(0.429cm)=(0,128,0);
        rgb255(0.571cm)=(255,255,0);
        rgb255(0.714cm)=(255,97,0);
        rgb255(0.857cm)=(106,0,0);
        rgb255(1cm)=(223,77,77);
    },
    enlargelimits=false,
    colorbar horizontal,
    point meta min=#3,
    point meta max=#4,
    axis line style = {draw=none},
    tick style = {draw=none},
    xtick = \empty, ytick = \empty,
    colorbar style={
        xlabel = {$\|\vec{\tau}_w\|$ [\si{\Pa}]},
        height=0.05*\pgfkeysvalueof{/pgfplots/parent axis height},
        width=0.9*\pgfkeysvalueof{/pgfplots/parent axis width},
        at={(0.5,-0.02)},
        anchor=center,
        tick label style={font=\footnotesize},
        #6
    },
    width=#5
]
    \addplotgraphicsnatural[xmin=0, xmax=1, ymin=0, ymax=1]{#2}

    \begin{scope}[xshift=0, yshift=100]
        \draw[->] #7 node[midway, anchor=#1] {$\vec{g}$};
    \end{scope}

    \begin{scope}[xshift=120, yshift=100]
        \draw[->, red] (0.1, 0.1) -- (0.09, 0.) node[pos=1, anchor=west] {$x$};
        \draw[->, green!60!black] (0.1, 0.1) -- (0.05, 0.04) node[pos=1, anchor=north east] {$y$};
        \draw[->, blue] (0.1, 0.1) -- (0.02, 0.13) node[pos=1, anchor=east] {$z$};
    \end{scope}

\end{axis}
\end{tikzpicture}
}


\begin{document}
\renewcommand{\thefootnote}{\fnsymbol{footnote}} % Use symbols globally


\maketitle

\footnotetext[1]{Cemosis, IRMA UMR 7501, Université de Strasbourg, CNRS, France}
\footnotetext[2]{University of Maine, Orono, ME, United States}
\footnotetext[3]{Université Paris Cité, CNRS, MAP5, F-75006 Paris, France}
\footnotetext[4]{Indiana University Bloomington, Bloomington, IN, USA}



\section*{Purpose}

Cell injection therapy is a promising treatment for Fuchs endothelial corneal dystrophy by delivering cultured human corneal endothelial cells into the anterior chamber (AC).
After injection, patients are rotated from a supine to a prone position for over three hours to promote sedimentation.
This study aims to develop a computational framework to understand how convection currents influence the wall shear stress (WSS), thereby informing strategies to optimize sedimentation and improve therapeutic outcomes.


\section*{Method}

A computational fluid dynamics framework investigated aqueous humor (AH) flow, focusing on velocity and WSS distributions.
The model governed by the Navier-Stokes equations with the Boussinesq approximation and gravitational effect account for heat exchange between the eye, surrounding tissues, and ambient environment.
Posture (supine and prone) and ambient temperature (\qty{15}{\degreeCelsius}, \qty{25}{\degreeCelsius}, and \qty{33}{\degreeCelsius}) were varied to assess their impact.
Elevated WSS may impede cell adhesion, while efficient mixing can enhance sedimentation and redistribution of injected cells.
AH secretion and outflow were neglected, as previous work (Ooi et al., \emph{Comput. Biol. Med.}, 2008) indicated buoyancy-driven flows dominate regardless of posture.
Simulations were performed using the \fpp{} framework (docs.feelpp.org), employing no-slip boundary conditions and solving for AH velocity and WSS distributions.


\section*{Results}

The simulations revealed that orientation and ambient temperature influence AH flows and WSS (\cref{fig:arvo_flows}).
Changes from supine to prone altered gravitational alignment, reducing AH velocities and affecting sedimentation.
At cooler ocular surface temperatures (\qty{15}{\degreeCelsius}), enhanced AH circulation increased WSS by up to \qty{83.4}{\percent}, promoting more uniform cell dispersion.
Warmer conditions (\qty{33}{\degreeCelsius}) reduced WSS by \qty{46}{\percent}, resulting in less uniform sedimentation patterns potentially hindering optimal cell adhesion (\cref{tab:arvo_wss}).


%!TeX root=../main.tex

\newcommand{\myplot}[4]{
    \begin{tikzpicture}
        \begin{axis}[
            colorbar,
            colormap={Rainbow Uniform}{
                rgb255(0.0cm)=(5,97,254);
                rgb255(0.023809523809523808cm)=(5,108,247);
                rgb255(0.047619047619047616cm)=(5,119,239);
                rgb255(0.07142857142857142cm)=(5,130,232);
                rgb255(0.09523809523809523cm)=(5,139,222);
                rgb255(0.11904761904761904cm)=(5,148,212);
                rgb255(0.14285714285714285cm)=(5,157,202);
                rgb255(0.16666666666666666cm)=(5,166,191);
                rgb255(0.19047619047619047cm)=(5,174,179);
                rgb255(0.21428571428571427cm)=(5,183,167);
                rgb255(0.23809523809523808cm)=(5,193,153);
                rgb255(0.2619047619047619cm)=(5,202,140);
                rgb255(0.2857142857142857cm)=(5,211,126);
                rgb255(0.30952380952380953cm)=(5,220,109);
                rgb255(0.3333333333333333cm)=(5,228,91);
                rgb255(0.35714285714285715cm)=(4,237,74);
                rgb255(0.38095238095238093cm)=(69,242,39);
                rgb255(0.40476190476190477cm)=(125,245,28);
                rgb255(0.42857142857142855cm)=(164,249,11);
                rgb255(0.4523809523809524cm)=(194,251,8);
                rgb255(0.47619047619047616cm)=(224,252,5);
                rgb255(0.5cm)=(254,254,3);
                rgb255(0.5238095238095238cm)=(254,243,20);
                rgb255(0.5476190476190477cm)=(254,232,37);
                rgb255(0.5714285714285714cm)=(254,220,55);
                rgb255(0.5952380952380952cm)=(254,208,55);
                rgb255(0.6190476190476191cm)=(254,196,55);
                rgb255(0.6428571428571429cm)=(254,183,55);
                rgb255(0.6666666666666666cm)=(254,171,55);
                rgb255(0.6904761904761905cm)=(254,159,55);
                rgb255(0.7142857142857143cm)=(254,147,55);
                rgb255(0.7380952380952381cm)=(254,132,55);
                rgb255(0.7619047619047619cm)=(254,118,55);
                rgb255(0.7857142857142857cm)=(254,104,55);
                rgb255(0.8095238095238095cm)=(253,84,53);
                rgb255(0.8333333333333334cm)=(251,66,48);
                rgb255(0.8571428571428571cm)=(252,37,53);
                rgb255(0.8809523809523809cm)=(242,29,64);
                rgb255(0.9047619047619048cm)=(230,20,74);
                rgb255(0.9285714285714286cm)=(218,10,84);
                rgb255(0.9523809523809523cm)=(203,11,91);
                rgb255(0.9761904761904762cm)=(189,11,98);
                rgb255(1.0cm)=(174,12,105);
            },
            enlargelimits=false,
            colorbar horizontal,
            point meta min=#2,
            point meta max=#3,
            axis line style = {draw=none},
            tick style = {draw=none},
            xtick = \empty, ytick = \empty,
            colorbar style={
                % xlabel style={
                %     at={(0.5,1.1)},
                %     anchor=south,
                % },
                xlabel = {$\vec{u}$ [\si{\meter\per\second}]},
                xlabel style={
                    yshift=1ex, % Adjust this value to reduce the space
                },
                height=0.05*\pgfkeysvalueof{/pgfplots/parent axis height},
                width=0.9*\pgfkeysvalueof{/pgfplots/parent axis width},
                at={(0.5,-0.02)},
                anchor=center,
                tick label style={
                    font=\footnotesize,
                    % yshift=0.5ex, % Adjust this value to reduce the space
                },
                x tick scale label style={yshift=1ex}
            },
            width=\subfigwidth
        ]
            \addplotgraphicsnatural[xmin=0, xmax=1, ymin=0, ymax=1]{#1}

            \draw[#4] (-.1, 0.8) -- (0.1, 0.8) node[midway, anchor=south] {$\vec{g}$};
            \draw[#4] (0.3, 0.5) -- (0.5, 0.5) node[midway, anchor=south] {$\vec{u}$};
        \end{axis}
    \end{tikzpicture}


}


\def\subfigwidth{0.5\textwidth}
\begin{figure}
    \centering
    \subfigure[Supine position with $T_\text{amb}=\SI{25}{\degreeCelsius}$.]{
        \myplot{flowsupine.png}{6.394491411422713e-9}{0.000025372790645038857}{->}
    }
    \subfigure[Prone position with $T_\text{amb}=\SI{25}{\degreeCelsius}$.]{
        \myplot{flowprone.png}{6.239064596468915e-9}{0.000015184686817409953}{<-}
    }

    \subfigure[Supine position with $T_\text{amb}=\SI{15}{\degreeCelsius}$.]{
        \myplot{flowsupine15.png}{8.426742526913508e-9}{0.000036873671901771015}{->}
    }
    \subfigure[Prone position with $T_\text{amb}=\SI{15}{\degreeCelsius}$.]{
        \myplot{flowprone15.png}{8.147417992892589e-9}{0.000018682715387970866}{<-}
    }

    \caption{AH flow in AC for various orientations and ambient temperatures.}
    \label{fig:arvo_flows}
\end{figure}


\begin{table}
    \centering
    \documentclass{standalone}
\usepackage{tabularray}
\usepackage{csquotes}
\usepackage[natbib, defernumbers=true, backend=biber, style=alphabetic, eprint=false, maxbibnames=99]{biblatex}
\addbibresource{biblio.bib}
\usepackage{pgfplots}
\pgfplotsset{compat=newest}
\usepackage{siunitx}
\newcommand{\shortcite}[1]{\citeauthor{#1} (\citeyear{#1}).}

\renewcommand{\thefootnote}{\fnsymbol{footnote}}

\begin{document}

\begin{tikzpicture}
    \node (tab) at (0,0){
        \begin{tblr}{colspec={|c|c|c|c|c|}}
        \hline
        \SetRow{azure7} \multirow{2}{*}{\textbf{Zone}} & \multirow{2}{*}{\textbf{Orientation}} & \multicolumn{3}{c|}{\textbf{Mean of magnitude of WSS}}\\
        \hline
        \SetRow{azure7}  &  & $T_\text{amb}= 25\si{\celsius}$ & $T_\text{amb}= 15\si{\celsius}$ & $T_\text{amb}= 33\si{\celsius}$\\
        \hline
        \SetRow{azure9} \multirow{2}{*}{Whole domain} & Prone  & \pgfmathprintnumber{6.40088e-07}~\si{\pascal} & $+72.96~\%$ & $-44.82~\%$ \\
        \SetRow{azure9}                               & Supine & \pgfmathprintnumber{6.77375e-07}~\si{\pascal} & $+81.20~\%$ & $-46.23~\%$ \\
        \hline
        \multirow{2}{*}{Cornea} & Prone  & \pgfmathprintnumber{2.30063e-07}~\si{\pascal} & $+71.63~\%$ & $-44.55~\%$ \\
                                & Supine & \pgfmathprintnumber{2.49599e-07}~\si{\pascal} & $+83.40~\%$ & $-46.89~\%$ \\
        \hline
        \SetRow{azure9} \multirow{2}{*}{Iris} & Prone  & \pgfmathprintnumber{3.17341e-07}~\si{\pascal} & $+73.67~\%$ & $-44.96~\%$ \\
        \SetRow{azure9}                       & Supine & \pgfmathprintnumber{3.30958e-07}~\si{\pascal} & $+79.76~\%$ & $-46.01~\%$ \\
        \hline
        \end{tblr}
    };

\end{tikzpicture}



\end{document}
    \caption{Evolution of the magnitude of the WSS for different zones of the eye on different regions, for various ambient temperature.}
    \label{tab:arvo_wss}
\end{table}






\section*{Conclusion}

Our findings highlight the crucial interplay of thermal and postural factors in shaping AH dynamics and mechanical stresses within the AC.
Cooler ocular surface conditions foster higher WSS and uniform cell dispersion, suggesting thermal modulation can improve endothelial cell therapy outcomes.
By adjusting body orientation and ocular surface temperature, clinicians may guide cell sedimentation and adhesion, enhancing postoperative results in anterior segment procedures.





\end{document}